\documentclass[12pt]{article}
\rmfamily
\usepackage[margin=1.0in]{geometry}

\title{Security of Real Systems Pentest Report}
\date{December 2020}
\author{Will Goodman}

\begin{document}
  \maketitle

  \section{Assess the System}
  I began reviewing the system by using nmap to perform port scans.
  The server was identified as a Debian (10 or greater) server with the following open ports:
  \begin{itemize}
    \item 22 - SSH
    \item 80 - HTTP (nginx)
    \item 443 - HTTPS (nginx)
  \end{itemize}
  In addition, the following ports are filtered:
  \begin{itemize}
    \item 135
    \item 139
  \end{itemize}

  \subsection{Port 22 - SSH}
  Ideally this port should only be open if it is really required.
  If it is not completely necessary, then disable SSH.

  Only public key encryption is enabled, which reduces the risks as brute-forcing passwords is not possible.
  It is imperative that the public keys are kept secure on the devices where they are stored.
  If encrypted hard-drives are not already part of the security policy, they should be encrypted as soon as possible.

  There were no known CVE vulnerabilities found on this implementation of SSH.

  \subsection{Port 80 - HTTP}
  Running the nmap http-enum script reveals no pages other than index.html.
  I would recommend adding a security.txt (https://tools.ietf.org/html/draft-foudil-securitytxt-10) containing contact details in case a third-party finds a vulnerability and wishes to make the CISO aware.
  Although there is currently no robots.txt, it is important not to list any admin or secure pages if one is ever added.

  An nmap vulnerability scan of the port revealed no known CSRF or XSS vulnerabilities.

  \subsection{Port 443 - HTTPS}
  The server only supports TLS 1.2 and 1.3, which is the best option at the moment as not all clients support TLS 1.2 yet.
  The server also prefers to use the more secure TLS 1.3, which is what I would recommend.
  The following cipher suites are supported.
  TLS 1.3:
  \begin{itemize}
    \item TLS\_AES\_256\_GCM\_SHA384
    \item TLS\_CHACHA20\_POLY1305\_SHA256
    \item TLS\_AES\_128\_GCM\_SHA256
  \end{itemize}
  TLS 1.2:
  \begin{itemize}
    \item TLS\_ECDHE\_RSA\_WITH\_AES\_128\_GCM\_SHA256
    \item TLS\_ECDHE\_RSA\_WITH\_AES\_128\_CBC\_SHA256
  \end{itemize}

  In general these cipher suites are deemed secure by modern standards, and achieve an A+ score on SSLLabs.
  That being said, I would recommend disabling TLS\_ECDHE\_RSA\_WITH\_AES\_128\_CBC\_SHA256 or changing it to the 256-bit AES version.
  CBC AES with a 128-bit key is now considered weak by modern standards.

  An nmap vulnerability scan of port 443 revealed no known CSRF or XSS vulnerabilities.
  However, the server appears to be vulnerable to a Denial-of-Service attack on an Apache server (CVE-2011-3192).
  If the Apache server is hosted behind nginx, I would recommend ensuring that Apache only listens on localhost (127.0.0.1) to reduce the attack surface.

  The certificate appears to be a Let's Encrypt certificate with a three month validity.
  I generally would recommend a longer certificate validity period (12 months), however 3 months is the maximum length for Let's Encrypt so that cannot change.

  \subsection{Ports 135/139}
  These ports are shown as filtered due to no response being received.
  If these are not required, then ensure they are closed.
  I have not been able to find any vulnerabilities or access any services on either of these ports, so if they are required they appear to be secure.

  \subsection{Summary}
  I recommend the following changes:
  \begin{itemize}
    \item Enforce encrypted hard-drives as standard
    \item Either remove the 128-bit AES CBC cipher suite or upgrade it to the 256-bit version
    \item Check if ports 135/139 are open. Close them if not needed
    \item Invest in some improved DoS protection
  \end{itemize}

  \section{Explain the System}

  The server has a few specific security features configured, which I explain in more detail in this section.

  \subsection{Host-Strict-Transport-Security}
  HSTS requires browsers to use HTTPS instead of HTTP for a set period of time.
  If a browser cannot ensure the security of the connection, for example if it does not trust the provided certificate, according to the specification it should terminate the connection.
  This helps prevent Man-in-the-Middle attacks by using an encrypted connection rather than plaintext which can be altered.
  On this server, HSTS is enabled with a period of 31,536,000 seconds (one year).
  One vulnerability of HSTS is if the initial connection, or the first connection after the period expires, is over HTTP.
  To help prevent this HSTS can be preloaded into browsers so that they know this server requires HTTPS before sending any requests.
  However, this server is not preloaded.
  The long expiration period reduces the risk, however for optimal security the server should be preloaded in browsers.

  \subsection{Online Certificate Status Protocol Stapling}
  OCSP Stapling allows a server to append (staple) a time-stamped OCSP response signed by a Certificate Authority to the initial TLS handshake.
  As the stapled certificate must be signed by the CA, they cannot be faked.
  As the certificate validation is performed by the certificate holder (server), the client does not need to reveal their browsing habits to a third-party, improving privacy.

  On this server, OCSP Stapling is enabled, and it appears the stapled certificate is updated on a weekly basis.
  However, the OCSP ``Must-Staple'' flag is not enabled.
  This flag, if enabled, tells browsers to drop the connection if no valid certificate is stapled, rather than reaching out to the OCSP server themselves.
  The privacy gain of enabling this flag is minimal, so this is not a major concern.

  \subsection{Content Security Policy}
  A Content Security policy ("content-security-policy" header) restricts how resources within the browser, such as JavaScript or CSS.
  It can help protect against Cross-Site-Scripting attacks.

  The server has the following content-security-policy header:
  \begin{verbatim}
  report-uri https://batten.report-uri.com/r/d/csp/enforce; 
  upgrade-insecure-requests; 
  default-src 'self'; 
  plugin-types application/pdf; 
  frame-ancestors 'self'; 
  img-src 'self' https:; 
  style-src 'self' https://www.batten.eu.org; 
  font-src 'self' data:
  \end{verbatim}

  This header provides improved security, although could be improved.

  The ``report-uri'' flag sends reports about violations of the policy to the given URL.
  This is very useful in a development environment when testing the CSP, however for a production website I would recommend removing it to ensure the CSP is enforced.

  The ``upgrade-insecure-requests'' flag tells the client to upgrade any HTTP link to HTTPS in case any legacy links exist.
  As this is a new website it would be best practice to make any links HTTPS anyway, however adding this flag acts as a safeguard.

  The ``plugin-types'' flag only allows the PDF plugin, and other plugins such as Flash/Java Applets will be blocked.
  This is much more secure than having no CSP, however I did not find any PDF files within the website.
  If no PDF files exist on the website, then this flag could be removed and replaced with an ``object-src `none' '' flag to block all plugins, which would be slightly more secure.

  The ``frame-ancestors'' and ``*-src'' flags tell the client where they can fetch resources from (up to two available sources).
  The `self' configuration tells the client that resources can only be collected from the same URL as the webpage, this is deemed secure.
  The `https:' flag allows images to be collected from any HTTPS URL.
  This is not as secure as a specified URL (such as in `style-src'), however is still deemed secure enough to be acceptable.
  The `font-src' flag has `data:' specified as a source, this is deemed insecure as an attacker could inject arbitrary data.
  This should be changed out for `https:' as a priority.

  \section{Extend the System}

\end{document}