\documentclass[12pt]{article}
\rmfamily
\usepackage[margin=1.0in]{geometry}

\title{Security of Real Systems Pentest Report}
\date{December 2020}
\author{Will Goodman}

\begin{document}
  \maketitle

  \section{Assess the System}
  I began reviewing the system by using nmap to perform port scans.
  The server was identified as a Debian (10 or greater) server with the following open ports:
  \begin{itemize}
    \item 22 - SSH
    \item 80 - HTTP (nginx)
    \item 443 - HTTPS (nginx)
  \end{itemize}
  In addition, the following ports are filtered:
  \begin{itemize}
    \item 135
    \item 139
  \end{itemize}

  \subsection{Port 22 - SSH}
  Ideally this port should only be open if it is really required.
  If it is not completely necessary, then disable SSH.

  Only public key encryption is enabled, which reduces the risks as brute-forcing passwords is not possible.
  It is imperative that the public keys are kept secure on the devices where they are stored.
  If encrypted hard-drives are not already part of the security policy, they should be encrypted as soon as possible.

  There were no known CVE vulnerabilities found on this implementation of SSH.

  \subsection{Port 80 - HTTP}
  Running the nmap http-enum script reveals no pages other than index.html.
  I would recommend adding a security.txt (https://tools.ietf.org/html/draft-foudil-securitytxt-10) containing contact details in case a third-party finds a vulnerability and wishes to make the CISO aware.
  Although there is currently no robots.txt, it is important not to list any admin or secure pages if one is ever added.

  An nmap vulnerability scan of the port revealed no known CSRF or XSS vulnerabilities.

  \subsection{Port 443 - HTTPS}
  The server only supports TLS 1.2 and 1.3, which is the best option at the moment as not all clients support TLS 1.2 yet.
  The server also prefers to use the more secure TLS 1.3, which is what I would recommend.
  The following cipher suites are supported.
  TLS 1.3:
  \begin{itemize}
    \item TLS\_AES\_256\_GCM\_SHA384
    \item TLS\_CHACHA20\_POLY1305\_SHA256
    \item TLS\_AES\_128\_GCM\_SHA256
  \end{itemize}
  TLS 1.2:
  \begin{itemize}
    \item TLS\_ECDHE\_RSA\_WITH\_AES\_128\_GCM\_SHA256
    \item TLS\_ECDHE\_RSA\_WITH\_AES\_128\_CBC\_SHA256
  \end{itemize}

  In general these cipher suites are deemed secure by modern standards, and achieve an A+ score on SSLLabs.
  That being said, I would recommend disabling TLS\_ECDHE\_RSA\_WITH\_AES\_128\_CBC\_SHA256 or changing it to the 256-bit AES version.
  CBC AES with a 128-bit key is now considered weak by modern standards.

  An nmap vulnerability scan of port 443 revealed no known CSRF or XSS vulnerabilities.
  However, the server appears to be vulnerable to a Denial-of-Service attack on an Apache server (CVE-2011-3192).
  If the Apache server is hosted behind nginx, I would recommend ensuring that Apache only listens on localhost (127.0.0.1) to reduce the attack surface.

  The certificate appears to be a Let's Encrypt certificate with a three month validity.
  I generally would recommend a longer certificate validity period (12 months), however 3 months is the maximum length for Let's Encrypt so that cannot change.

  \subsection{Ports 135/139}
  These ports are shown as filtered due to no response being received.
  If these are not required, then ensure they are closed.
  I have not been able to find any vulnerabilities or access any services on either of these ports, so if they are required they appear to be secure.

  \subsection{Summary}
  I recommend the following changes:
  \begin{itemize}
    \item Enforce encrypted hard-drives as standard
    \item Either remove the 128-bit AES CBC cipher suite or upgrade it to the 256-bit version
    \item Check if ports 135/139 are open. Close them if not needed
    \item Invest in some improved DoS protection
  \end{itemize}

  \section{Explain the System}

  \section{Extend the System}
\end{document}